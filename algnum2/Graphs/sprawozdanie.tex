\documentclass{article}
\usepackage{graphicx} 
\usepackage{amsmath}

\title{Projekt 2}
\author{Wojtek Balcer, Michał Safuryn, Bartek Smolibowski}
\date{April 2024}

\begin{document}

\maketitle
\section{Opis}

\noindent
\textbf{Zależności Projektu} \\
\begin{tabular}{>{\bfseries}rl}
Rozszerzenia: & $R_0$ and $R_1$ \\
Użyte języki: & Java, Python, Bash \\
Użyta struktura: & $DS_1$ \\
Sposób przechowywania macierzy: & Mapa map i intów dla niezerowych wartości \\
\end{tabular}

\clearpage

\section{Monte Carlo}

\hspace{}

\textbf{Wykresy różnicy pomiedzy testami}

\begin{figure}[ht!]
    \centering
    \includegraphics[width=1\linewidth]{comparison_plots.png}
    \label{fig:my_label}
\end{figure}



\clearpage

\section{Hipoteza  $H_1$}
\textbf{$H_1$:} Algorytm  $A_2$ zwykle daje dokładniejsze wyniki niż $A_1$. Różnica dokładności rośnie wraz z rozmiarem macierzy i liczbą niezerowych współczynników.

\hspace{}

\textbf{Wykresy różnicy $A_1$ - $A_2$}

\text{Wyniki są identyczne oprocz testu gdzie różnica jest na dopusczalnym poziomie}
\textbf{Hipoteza nieprawdziwa.} 
\clearpage

\begin{figure}[ht!]
    \centering
    \includegraphics[width=1\linewidth]{h1_plot_7.jpg}
    \label{fig:my_label}
\end{figure}

\begin{figure}[ht!]
    \centering
    \includegraphics[width=1\linewidth]{h1_plot_8.jpg}
    \label{fig:my_label}
\end{figure}

\begin{figure}[ht!]
    \centering
    \includegraphics[width=1\linewidth]{h1_plot_9.jpg}
    \label{fig:my_label}
\end{figure}




\clearpage

\section{Hipoteza $H_2$}

\textbf{Algorytm $A_3$ działa dla postawionego zadania.}

\hspace{}

\textbf{Wykresy różnicy $Monte$ - $A_3$}

\begin{figure}[ht!]
    \centering
    \includegraphics[width=1\linewidth]{a3.png}
    \label{fig:my_label}
\end{figure}




\textbf{Hipoteza prawdziwa. Róznica pomiędzy wynikami jest  dopuszczalna oscylująca na tysięcznych procenta.} 


\clearpage

\section{Hipoteza $h_3$}

\textbf{$H_3$: Jeśli algorytm $A_3$ jest zbieżny do rozwiązania, to wyniki otrzymujemy istotnie szybciej niż dla $A_1$ i $A_2$}

\begin{figure}[ht!]
    \centering
    \includegraphics[width=1\linewidth]{czasy.jpg}
    \label{fig:enter-label}
\end{figure}



\textbf{Hipoteza nieprawdziwa.} 

\end{document}